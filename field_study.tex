\section{China Customs} 
China Customs систем нь Хятад улсын хилээр нэвтэрж буй жуулчид болон иргэдээс эрүүл мэндийн судалгаа авдаг систем юм. China customs хувьд хэрэглэгчийн хувийн мэдээлэл, тухайн хүний хаана суурьших болон корона вирусын шинж тэмдэг илэрсэн эсэх, коронавирусын шинжилгээний хариу зэрэг асуултыг асуудаг. Энэхүү системийн хувьд асуултын хэсэг нь нэг л нүүр байх бөгөөд шинж тэмдэгүүдүүдийг нэгтгэн нэг л асуулт болгосон нь хэрэглэхэд илүү хялбар болгосон давуу талтай. 

\section{Korea Customs Service} 
Korea Customs Service нь Солонгос улсын хилээр нэвтэрж буй хүмүүсээр эрүүл мэндийн асуулга бөглүүлэх систем юм. Энэхүү системийн хувьд асуулга бөглөхөөс өмнө хэрэглэгчийн хувийн мэдээллийг асуух ба тухайн мэдээлэл нь хүчинтэй тохиолдолд үргэлжлүүлэх боломжтой болдог.
Ингэснээр асуулга бөглөх хэрэглэгчийн мэдээлэл хүчинтэй байна гэдэг нь энэхүү системийн давуу тал юм. 

\section{Arrive MN-тэй харьцуулах нь}
Дээрх системүүдийн адилаар Arrive MN нь хэрэглэгч эрүүл мэндийн асуулга бөглөх тухайн хариултыг хянах зорилго бүхий систем билээ. Arrive MN нь хэрэглэгчийн бүртгэлийг тусд нь үүсгэх ба Монгол хүний хувьд зөвхөн регистерийн дугаараар бусад мэдээллийг нь татан ашигладаг тул бүртгэлийг хялбар хийх давуу талтай. Arrive MN дээрх асуултууд нь динамик буюу асуулт болон сонголтыг өгөгдлийн сангаас аван ашигладаг нь асуултуудыг хялбар өөрчлөх боломж олгодог. 
